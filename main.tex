
\documentclass[conference]{IEEEtran}

\usepackage{rotating}
\usepackage{ifpdf}
\usepackage{algorithmic}
\usepackage{graphicx}
\usepackage{multirow}
\usepackage[TABBOTCAP]{subfigure}
\usepackage{caption}
\usepackage{subfigure}
\usepackage{amsfonts}

\usepackage{latexsym}
\usepackage{amssymb}
\setcounter{tocdepth}{3}
\usepackage{lscape}
\usepackage[usenames]{color}


\ifCLASSINFOpdf


\fi


\hyphenation{op-tical net-works semi-conduc-tor}


\begin{document}

\title{GoBus: Progettazione e Sviluppo di un\rq App a Supporto della Mobilit\`{a}}

\author{\IEEEauthorblockN{Gemma Catolino, Elisa D\rq Eugenio, Davide De Chiara, Alessandra Longo\\Dipartimento di Studi e Ricerche Aziendali -  Management \& Information Technology\\Universit\`{a} degli Studi di Salerno, Fisciano (SA), Italia\\\emph{Tutors:} Pasquale Salza, Andrea De Lucia, Filomena Ferucci}}


\maketitle


\begin{abstract}
Il problema della mobilit\`{a} rappresenta una delle maggiori sfide per i governi di tutto il mondo, a causa dei problemi sempre crescenti che riguardano l\rq inquinamento acustico ed ambientale. Nel corso degli ultimi anni, l\rq Unione Europea ha incentivato, tramite campagne promozionali e stanziamenti di fondi ad hoc, l\rq utilizzo di metodi alternativi per la circolazione delle persone. In questo contesto, i problemi organizzativi legati all\rq utilizzo dei trasporti pubblici nelle citt\`{a}, grandi o piccole che siano, sono particolarmente sentiti. Le informazioni circa orari, corse, linee e fermate non sempre sono facilmente reperibili da parte di chi si trova ad usufruire dei servizi di trasporto. La naturale conseguenza \`{e} rappresentata da disagi di ogni sorta di cui quotidianamente si parla sui mass media. In questo documento viene presentato \emph{GoBus}, un software per la semplificazione delle attivit\`{a} di reperimento delle informazioni da parte dei cittadini. Progettata come un\rq applicazione consultabile sia tramite browser web che come applicazione mobile, \emph{GoBus} consente agli utenti di poter accedere alle informazioni sulla mobilit\`{a} organizzata dai servizi di trasporto di ogni citt\`{a}. L\rq ausilio delle tecniche e dei processi proposti dall\rq Ingegneria del Software ha consentito la realizzazione di un software conforme ai reali bisogni dei cittadini. Inoltre, lo svolgimento del progetto ha seguito le linee guida riguardo la gestione dei progetti software. A supporto di questa idea, \`{e} stata realizzata un\rq applicazione web, con funzione di prototipo, che dimostra la reale fattibilit\`{a} dei requisiti identificati e che illustra un suo possibile utilizzo in un ambiente reale.\\
URL: http://tinyurl.com/kareo9e
\end{abstract}

\IEEEpeerreviewmaketitle



\section{Introduzione}
\label{intro}
Il problema della mobilit\`{a} urbana

\section{Problem Statement}
\label{PS}
La maggior parte delle citt\`{a}, grandi o medie che siano, posseggono una fitta rete di collegamenti. Tali collegamenti il più delle volte appaiono difficili da comprendere e creano spesso confusione sia non solo ai visitatori occasionali, ma anche alle persone più pratiche della citt\`{a}. Nel corso degli anni, il progresso tecnologico ha fatto in modo che fossero sviluppate applicazioni mobile a supporto della fruibilit\`{a} delle informazioni sui collegamenti che le aziende di trasporto forniscono. 
Da allora i più famosi store per smartphone (App Store\footnote{https://itunes.apple.com/us/genre/ios/id36?mt=8}, Google Play \footnote{https://play.google.com/store}, Windows Phone\footnote{http://www.windowsphone.com/it-it/store}) rendono disponibili ogni giorno e con continui aggiornamenti applicazioni mobile, ma queste, per la maggior parte, sono legate solamente alla specifica città e non orientate verso un concetto più ampio come l’intera nazione.\\ 
Avere un’applicazione di carattere nazionale consentirebbe a tutti i cittadini o turisti di essere maggiormente informati su tutta la mobilit\`{a} italiana. Risulterrebbe inoltre pi\`{u} pratica dato che in una sola \emph{app} potrebbero essere consultati e/o programmati orari, linee e percorsi di tutte le corse urbane della propria città o a seconda di dove ci si trova.\\
\emph{Moovit} \footnote{https://itunes.apple.com/it/app/moovit-trasporto-pubblico/id498477945?mt=8} (disponibile per i tre principali store) è una delle poche applicazioni che comprende una vasta scelta di città (all'incirca 20). Rende non solo disponibile informazioni di collegamento riguardo autobus ma anche di treni e metro (qualora presenti). L'applicazione consente di ricercare e di fruire informazioni riguardo ogni tipo di collegamento e pianificare un eventuale viaggio urbano. 
La caratteristica più interessante di questa applicazione è che risulta molto social. Essa infatti reperisce informazioni sulla qualità del percorso della linea di un collegamento (affollamento, orario di arrivo) tramite le valutazione degli utenti. 
Nel suo complesso \emph{Moovit} risulta essere completa e molto utile, in più viene resa appetibile grazie al suo aspetto social, come sopra citato. Tuttavia presenta dei limiti. Navigandola infatti, si mostra inizialmente abbastanza confusionaria in quanto risulta complicato capire come poter usufruire delle funzionalità che essa mette a disposizione. Un altro limite risiede nella sola possibilità di basarsi sulla posizione attuale, escludendo quindi la possibilità di selezionare una diversa città, qualora si voglia prevenire futuri spostamenti in altre metropoli.
Inoltre è ovvio pensare che il servizio \emph{``Social''} (ciliegina sulla torta  dell'applicazione) non potrà essere garantito per sempre, poichè troppo legato all'attività degli utenti.\\
Da questa problematica nasce l’idea di sviluppare \emph{GoBus}, un’applicazione pensata per essere accessibile sia tramite una web application che da dispositivi mobile (ad esempio, Windows Phone).\\
\emph{GoBus} consentirà di avere a disposizione tutti i collegamenti delle citt\`{a} italiane, a seconda di dove ci si trova. Consentirà di dare informazioni sui percorsi da seguire, e in base alla propria posizione visualizzare fermate e linee nelle vicinanze, oppure scegliere comodante da soli una citt\`{a}.
Tutto questo grazie anche alle aziende di trasporti (oltre agli admin dell’applicazione) le quali potranno caricare i dettagli appartenenti ai collegamenti e le varie news associate. Tali file che le aziende caricheranno non saranno lavoro aggiuntivo per loro, dato che si utilizzeranno gli stessi file che le aziende forniscono a Google.
\section{Management}
\label{management}
Dal punto di vista organizzativo del progetto \emph{GoBus}, si è deciso grazie alle esperienze acquisite dal team di cercare di trovare un giusto equilibrio tra il ciclo di vita di un progetto, costituito dalle fasi di ingegnerizzazione e produzione, e il ciclo gestionale costituito invece dalle attività e processi svolti dal team per governare il progetto. Quindi questa fusione di conoscenze ci ha permesso di gestire al meglio il ciclo di vita del progetto, si è deciso di seguire le linee guida dettate dal \emph{Project Management Body of Knowledge} che descrive l’insieme delle prassi standard per la gestione di progetti così come definite dal “Project Management Institute”. I processi del Project Management descritti nel \emph{PMBOK®} \cite{PMBOK} sono 47 distribuiti in 5 gruppi, in particolare nell’ambito di GoBus sono stati utilizzati per:

\begin{itemize}
	\item {\bf{Inizializzazione}}: Raggruppa i processi necessari a selezionare un progetto in relazione a specifici obiettivi di business e definendo obiettivi e modalità di gestione del progetto.
	\item {\bf{Pianificazione}}: Finalizzata a circoscrivere l’ambito ed i deliverables del progetto, a definire requisiti di ciascun deliverable e a definire un piano di Project Management contenente tutti i vari piani in dettaglio per la gestione di: tempi, risorse, costi, qualità, rischi, comunicazione.
	\item {\bf{Esecuzione}}: Finalizzata a gestire e sviluppare il progetto, produrre i deliverables concordati.
	\item  {\bf{Monitoraggio e Controllo}}: Servono a valutare l’avanzamento dei lavori, a gestire eventuali modifiche e a verificare la qualità di quanto realizzato.
	\item {\bf{Chiusura}}: Finalizzata a gestire la chiusura del progetto.
\end{itemize}

\subsection{Inizializzazione}
In questa fase è stato inizializzato il progetto puntando prima a definire il team e identificare gli stakeholder di progetto e successivamente a definire Business need e strategia da attuare. Per quanto riguarda il team è stato creato come deliverable il team contract dove sono state definite alcune regole che riguardano comunication e people management, mentre per gli stakeholder è stato creato un registro dove memorizzare i vari stakeholder del progetto nel tempo e i loro contatti. Dopo aver creato le basi del team di progetto si è definito come lo Statement of work che racchiude in se le caratteristiche che dovrà avere il progetto, ovvero le Business needs, le funzionalità chiave e il piano strategico che si punta ad ottenere con GoBus. Per quanto riguarda i Bussiness needs come definito anche da ITIL® \cite{ITIL}, \emph{``ogni servizio deve avere una giustificazione''} per questo motivo all’interno di un progetto è fondamentale identificarli e renderli protagonisti della strategia di sviluppo del progetto. Tra i Business needs citati nello Statement of work abbiamo ad esempio la fruibilità dell’informazione, vitale è la knowledge sharing, ovvero la diffusione della conoscenza e informazione tramite il servizio offerto. Dallo Statement of work è già possibile iniziare a tirar fuori i requisiti funzionali che saranno successivamente identificati meglio all’interno del Requirements Analysis Document.

\subsection{Pianficazione}
TODO

\subsection{Esecuzione}
TODO

\subsection{Monitoraggio e Controllo}
Il monitoraggio e controllo è una fase che ha inizio insieme alla fase di pianificazione e dura fino alla fase di chiusura, è una fase dove si valuta l’avanzamento dei lavori e si attua il piano che è stato concepito nel Verification \& Validation Plan seguendo le metriche di misurazione di qualità definite nel Quality Plan. In pratica durante le fasi di pianificazione e esecuzione tramite l’utilizzo di checklist definite nel V \& V si revisionano i documenti creati. Vi sono diverse tipologie di checklist quelle per i documenti, dove vi è un componente del team che compila le checklist e quando termina invia le checklist al creatore del documento che la controlla e decide di accettarla e apportare le modifiche richieste. Di seguito il template utilizzato per la checklist di revisione dei documenti:\\

FIGURA\\\\

 Altre tipologie di checklist che abbiamo definito sono per la raccolta e analisi dei requisiti, stima dei costi e dei tempi, software configuration management, risk management, quality assurance e  testing. La compilazione di queste checklist da parte del team porta a una chiara misurazione della qualità del progetto, per maggiori dettagli sulle checklist invito a consultare il documento Verification \& Validation Plan. Dopo questa analisi che ci ha permesso di monitorare il progetto sono stati redatti dei management e quality report per capire appunto cosa doveva essere revisionato e aggiornato, un esempio nel quality report:\\
 
 FIGURA\\\\

Dove B sta a indicare che il documento è stato accettato ma vanno apportate lievi modifiche.

\subsection{Chiusura}
Una volta che arriveremo alla fase di chiusura di progetto e quindi aver ottenuto il risultato sulla base delle specifiche date in input, si procederà con la stesura della post-mortem review ovvero un deliverable per analizzare gli elementi del progetto e capire se sono stati di successo o di insuccesso. Template che sarà utilizzato per la post-mortem Review:\\

FIGURA\\\\

Dopo aver dato il proprio giudizio vi sono una serie di domande aperte che riguardano lesson learned e considerazioni finali.
La post-mortem review ci aiuterà nel gestire futuri rischi già incotrati in questo progetto quindi migliorando le nostre skill dal punto di vista del risk management e delle Best Practice.



\section{Software}
\label{software}
Questa sezione descrive il processo di sviluppo adottato nell\rq ambito del profetto \emph{GoBus}.

\subsection{Modello del Ciclo di Vita}
In base alle esigenze relative alle tempistiche e agli obiettivi del progetto, il modello del ciclo di vita che meglio si adatta \`{e} il ciclo \emph{evolutivo a prototipazione}. In particolare, vista la necessita di comprendere meglio i requisiti che il committente richiedeva, la tecnica utilizzata consente di fornirgli un prototipo del sistema in tempi rapidi. In questo modo, \`{e} possibile periodicamente avere feedback utili per la continuazione delle attivit\`{a} progettuali.

\subsection{Requirement Elicitation and Analysis}
La fase di requirement elicitation consiste nella comprensione delle necessit\`{a} del committente, al fine di collezionare i requisiti del sistema. Nell\rq ambito del progetto, tale attivit\`{a} \`{e} stata svolta facendo brainstorming con tutti gli stakeholder interessati. In particolare, sono stati organizzati incontri sia con i coordinatori del progetto sia con aziende di trasporto locali. Per mitigare il richio di errata comprensione delle reali necessit\`{a} degli stakeholder, i meeting si sono ripetuti cos\`{i} che, ad ogni incontro, i requisiti potessero essere meglio fissati. Il numero totale di incontri con gli stakeholder \`{e} 3.\\
La tabella \ref{tab:gestioni} mostra, ad alto livello, i principali raggruppamenti delle funzionalit\`{a} identificate nell'ambito del progetto \emph{GoBus}. Di seguito verr\`{a} analizzato in dettaglio ciascun raggruppamento. Si noti che per evitare ridondanza e per ragioni di spazio, non sono riportati tutti i requisiti funzionali identificati. Per la lista completa, si faccia riferimento al Documento di Analisi dei Requisiti (RAD) in allegato.\\

\begin{table*}[tb]
   \centering
   \caption{Overview delle Funzionalit\`{a} Identificate}
   \label{tab:gestioni}
  \resizebox{1\linewidth}{!}{
   \begin{tabular}{ll}\hline
   Nome & Descrizione\\\hline
   Gestione Registrazione & Insieme di funzionalit\`{a} che consentono la registrazione di un nuovo utente.\\
   Gestione Autenticazione & Insieme di funzionalit\`{a} che consente il riconoscimento degli utenti registrati.\\
   Gestione Account & Insieme di funzionalit\`{a} per la manipolazione degli account degli utenti.\\
   Gestione Fermate & Funzionalit\`{a} che consente di visualizzare le fermate tramite l\rq applicazione di diversi filtri per la ricerca mirata.\\
   Gestione Linee & Funzionalit\`{a} che consente la visualizzazione delle linee tramite l\rq applicazione di diversi filtri per la ricerca mirata.\\
   Gestione Percorso & Insieme di funzionalit\`{a} che consente la visualizzazione delle indicazioni stradali.\\
   Gestione Preferiti & Funzionalit\`{a} che consente la gestione delle fermate e delle linee preferite.\\
   Gestione News & Insieme di funzionalit\`{a} per la efficace gestione degli avvisi da parte delle aziende di trasporti.\\
   Gestione Dati GTFS & Insieme di funzionalit\`{a} che consente ad un\rq azienda di trasporti di caricare il proprio file GTFS.\\
   \hline
   \end{tabular}
   }
\end{table*}

\noindent {\bf{Gestione Registrazione}}: La gestione della registrazione da la possibilit\`{a} ad un nuovo utente di potersi iscrivere alla piattaforma. Vale la pena notare che la registrazione \`{e} consentita i) utilizzando la propria e-mail, fornendo un nickname ed una password; ii) tramite l'utilizzo dell'account Facebook; iii) tramite l'utilizzo dell'account Google. Questo insieme di requisiti \`{e} comune sia alla parte relativa alla web application che all'applicazione mobile.
\noindent {\bf{Gestione Autenticazione}}: Questa gestione permette ad un utente registrato di essere riconosciuto e abilitato a effettuare determinate operazioni da parte del sistema. Quindi, le funzionalit\`{a} di login e logout fanno parte di questa gestione.\\

\noindent {\bf{Gestione Account}}: Questo insieme di funzionalit\`{a} si occupa di tutto ci\`{o} che ha a che fare con la manipolazione delle informazioni personali degli utenti registrati al sistema. Nel dettaglio, \`{e} possibile visualizzare, modificare ed eliminare un account dalla piattaforma.\\

\noindent {\bf{Gestione Fermate}}: Questa funzionalit\`{a} permette di visualizzare le fermate tramite l\rq applicazione di diversi filtri, al fine di effettuare una ricerca mirata delle informazioni di cui l\rq utente necessita. I requisiti funzionali identificati riguardano la visualizzazione delle fermate tramite i) geo-localizzazione della posizione dell'utente; ii) una localit\`{a} specifica; o iii) nome della localit\`{a}. Inoltre, \`{e} possibile visualizzare le linee relative ad una fermata, consentendo quindi di conoscere, data una fermata, quali linee la percorrono.\\

\noindent {\bf{Gestione Linee}}: Questa funzionalit\`{a} permette di visualizzare le linee tramite l\rq applicazione di diversi filtri per effettuare una ricerca mirata delle informazioni di cui l\rq utente necessita. Sar\`{a} consentita la visualizzazione delle: linee di una specifica localit\`{a}, ma anche la ricerca di una linea e la visualizzazione di tutte le corse di una linea con i relativi orari di andata e ritorno. Inoltre, sar\`{a} implementata anche la visualizzazione dell'intero percorso di una determinata corsa.\\

\noindent {\bf{Gestione Percorso}}: Questa funzionalit\`{a} da all'utente la possibilit\`{a} di richiedere la navigazione tra un punto di partenza ed uno di arrivo.\\

\noindent {\bf{Gestione Preferiti}}: Tramite questa funzionalit\`{a} l'utente avr\`{a} la possibilit\`{a} di memorizzare le sue corse, linee, e citt\`{a} preferite, cos\`{i} da poter rapidamente accedere a tali informazioni in futuro.\\

\noindent {\bf{Gestione News}}: Questa funzionalit\`{a} \`{e} implementata esclusivamente per la parte riguardante la web application e consente ad una azienda di trasporti di poter inserire, modificare ed eliminare avvisi.\\

\noindent {\bf{Gestione dati GTFS}}: Questa funzionalit\`{a} \`{e} implementata esclusivamente per la parte riguardante la web application e consente ad una azienda di trasporti di poter caricare il proprio file GTFS.

L'analisi dei requisiti \`{e} poi proceduta tramite la modellazione delle funzionalit\`{a} identificate in casi d\rq uso. In questo documento, verr\`{a}  proposto un solo caso d\rq uso, cos\`{i} da consentire al lettore di poter comprendere la metodologia di modellazione adottata. In particolare, verr\`{a}  seguita la modellazione del requisito riguardante la ricerca di una linea. 

\begin{figure*}[tb]
\centering
\includegraphics[scale=.7]{img/cd.png}
\caption{Caso d\rq uso relativo al requisito funzionale che consente la ricerca di una linea.}
\label{fig:cd}
\end{figure*}

La Figura \ref{fig:cd} mostra il caso d\rq uso relativo al requisito funzionale che consente la ricerca di una linea. In particolare, l'utente registrato, dopo aver effettuato l'accesso al sistema (condizione d\rq entrata necessaria), clicca sul tasto che consente la ricerca di una linea e digita la query di ricerca prescelta. Il sistema acceder\`{a} all'archivio dati e caricher\`{a} la linea ricercata. Vale la pena notare che ogni caso d\rq uso del sistema modella un requisito sia considerando le azioni che intercorrono tra l\rq utente e il sistema, sia considerando i requisiti di qualit\`{a} ed eventuali condizioni che potrebbero causare eccezioni.\\

Il successivo step nella modellazione delle funzionalit\`{a} del sistema \`{e} consistito nella definizione dei path navigazionali relativi a ciascun raggruppamento di requisiti identificato. Di seguito \`{e} mostrato il path navigazionale che consente ad un utente registrato di muoversi all\rq interno della gestione delle linee, di cui il requisito di ricerca delle linee \`{e} incluso.

\begin{figure}[!h]
\centering
\includegraphics[scale=.4]{img/np.png}
\caption{Path navigazionale relativo alla gestione delle linee.}
\label{fig:np}
\end{figure} 

Come mostrato in Figura \ref{fig:np}, l'utente, a partire dalla Home Page dell'applicazione, pu\`{o} accedere alla funzionalit\`{a} di Gestione delle Linee, all\rq interno della quale \`{e} data la possibilit\`{a} di accedere i) alla visualizzazione delle linee tramite geo-localizzazione e  ii) visualizzazione della ricerca di una linea. Effettuando il caso d\rq uso mostrato in Figura \ref{fig:cd}, l\rq utente accede alla schermata di visualizzazione delle corse di andata per la linea selezionata, con la possibilit\`{a} di visualizzare anche la corsa di ritorno.\\

Infine, l'ultima parte dell\rq analisi dei requisiti \`{e} consistita nella definizione dei mockup delle funzionalit\`{a} identificate. In Figura \ref{fig:ui} \`{e} mostrato il mockup della funzionalit\`{a} di ricerca di una linea, implementato sulla base di una interfaccia per smartphone. 

\begin{figure}[h]
\centering
\includegraphics[scale=.3]{img/ui.png}
\caption{Screen Mockup della Funzionalit\`{a} di Ricerca di una Linea.}
\label{fig:ui}
\end{figure} 

La definizione degli screen mockup ci ha aiutato nella comprensione dei requisiti, ma anche nella definizione degli artefatti di basso livello descritti nelle successive sezioni. 

\subsection{Design}
Gli obiettivi principali della fase di design sono stati quelli di definire i) i design goal; ii) l\rq architettura dell\rq applicazione; e iii) il mapping hardware-software. Il primo step \`{e} consistito nella definizione degli obiettivi di design a partire dai requisiti non funzionali presenti nel documento di analisi e specifica dei requisiti. Sono stati identificate 4 categorie di design goal, di seguito riportati:\\

\noindent {\bf{DG-0 - Dependability criteria}}:  \emph{GoBus} garantir\`{a} il corretto svolgimento delle proprie funzioni, gestendo i vari errori logici (quelli derivanti da una negligenza da parte dell\rq utente), che potranno verificarsi durante l\rq utilizzo, ed eventuali attacchi alla sicurezza. Questo insieme di design goal comprende i requisiti di robustezza, affidabilit\`{a}, disponibilit\`{a} e sicurezza.\\
\noindent {\bf{DG-1 - Performance criteria}}:  Il sistema sar\`{a} usabile e leggero, in modo tale che, nel caso in cui più persone accedano al sistema contemporaneamente, questo non venga rallentato. In definitiva, il sistema dovr\`{a} garantire che le varie operazioni offerte vengano svolte entro un intervallo di tempo accettabile. Questo insieme di design goal comprende i tempi di risposta, il throughput, e i requisiti di memoria.\\
\noindent {\bf{DG-2 - Maintenance criteria}}:  \emph{GoBus} garantir\`{a} un alto grado di manutenibilit\`{a}. Questo insieme di design goal comprende l\rq estendibilit\`{a}, la modificabilit\`{a} e la tracciabilit\`{a} dei requisiti.\\
\noindent {\bf{DG-3 - End User criteria}}:  \emph{GoBus} garantir\`{a} la learnability e l\rq usabilit\`{a} del sistema.\\

Per quanto riguarda l\rq architettura del sistema, \emph{GoBus} sar\`{a} implementato in una architettura three-tier, e quindi tramite una suddivisione del sistema in tre livelli: i) \emph{presentation-layer}, \emph{application-layer} e \emph{storage-layer}. Il primo livello consiste di tutte le interfacce grafiche che consentiranno all\rq utente di poter dialogare con la logica di business, a sua volta implementata nell\rq application-layer. Infine, lo storage-layer contiene le componenti che consentono l\rq immagazzinamento dei dati.\\

Il successivo passo \`{e} consistito nella decomposizione del sistema in sottosistemi. In questo contesto, le metriche di coesione e accoppiamento rappresentano un importante strumento per poter definire una decomposizione che consenta a team di sviluppo separati di poter eseguire parallelamente diversi task. Nel nostro contesto, \`{e} stata definita la decomposizione mostrata in Figura \ref{fig:sd}.

\ref{fig:sd}
\begin{figure*}[tb]
\centering
\includegraphics[scale=.4]{img/sd.png}
\caption{Decomposizione in Sottosistemi}
\label{fig:sd}
\end{figure*} 

Come \`{e} possibile vedere, \emph{GoBus} si compone di tre sottosistemi, che fanno riferimento i) alla gestione degli utenti (Gestione Account in Figura \ref{fig:sd}), ii) la gestione di tutti i dati relativi alle fermate, alle linee e ai percorsi (Gestione Collegamenti in Figura \ref{fig:sd}), e iii) la gestione dei preferiti, che rappresenta una componente a se stante nell\rq economia del sistema.\\

\begin{figure*}[tb]
\centering
\includegraphics[scale=.4]{img/mhs.png}
\caption{Mapping Hardware/Software}
\label{fig:mhs}
\end{figure*} 

Infine, \`{e} stato definito il mapping hardware-software. Il Deployment Diagram fornisce un ausilio agli sviluppatori per quanto riguarda l\rq organizzazione delle componenti hardware e software del sistema \emph{GoBus}. In figura \ref{fig:mhs} possiamo vedere quali sono i nodi che interagiscono col sistema: Application Server e il Database Server. Le interfacce dei vari sottosistemi accedono ai pacchetti dell\rq Application Server, in cui risiedono gli oggetti di tipo control ed entity. L\rq accesso al database avviene tramite un sottosistema di storage, per garantire, anche, l\rq indipendenza di esso; infatti se sorgesse la necessit\`{a} di modificare una qualsiasi componente dell\rq interfaccia del sottosistema, non vi sarebbe il bisogno di apportare innumerevoli modifiche all\rq interno del sistema. La comunicazione tra i nodi avviene tramite protocollo HTTP.\\
Per quanto riguarda la base di dati del progetto \`{e} stata modellata sui file in formato General Transit Feed Specification Reference (GTFS). Questo formato \`{e} attualmente utilizzato dalle aziende di trasporto pubblico per la comunicazione dei dati, 
relativi ai servizi forniti, all\rq applicazione Google Transit. Questo consente alle aziende di 
fornire informazioni al progetto in modo facile e veloce, senza ridondanza di dati. 
Google fornisce le linee guida per la formattaione dei dati in questione, identificando un 
insieme di file. Tra questi ci sono file richiesti e file opzionali, ogni agenzia pu\`{o} decidere 
quali informazioni fornire in base alle proprie esigenze. \emph{GoBus} si propone di raggruppare 
in un\rq unica applicazione informazioni su tutte le agenzie di trasporto nazionali, per cui 
gestisce solo i dati obbligatori pi\`{u} alcune informazioni che non possono essere ignorate 
se fornite dall\rq agenzia, come ad esempio le eccezioni al calendario delle corse in giorni 
particolari. Questi file sono:\\

\begin{itemize}
\item {\bf{agency.txt}}: Uno o pi\`{u} agenzie di transito che forniscono i dati.\\
\item {\bf{stops.txt}}: Luoghi dove i bus fanno salire o scendere i passeggeri.\\
\item {\bf{routes.txt}}: Un percorso \`{e} un gruppo di viaggi che vengono visualizzati come un unico servizio.\\
\item {\bf{trips.txt}}: Viaggi per ogni itinerario. Un viaggio \`{e} una sequenza di due o pi\`{u} ferma relative ad un viaggio.\\
\item {\bf{stop\_ times.txt}}: Orario in cui un veicolo arriva ad una fermata, per ogni viaggio.\\
\item {\bf{calendar.txt}}: Date per i servizi che utilizzano un programma settimanale. Specifica quando viene avviato il servizio e finisce, oltre ai giorni della settimana in cui il servizio\`{e} disponibile.\\
\item {\bf{calendar\_ dates.txt}}: Date particolari in cui alcuni servizi subiscono una variazione.\\
\end{itemize}
All\rq interno degli stessi file, sono indicati dei campi obbligatori e dei campi opzionali.
Per comprendere la struttura della base di dati la \emph{figura 17} mostra il Diagramma Entit\`{a} Relazioni.



\begin{figure}[h]
\centering
\includegraphics[scale=.4]{img/ER.png}
\caption{Diagramma Base di Dati}
\label{fig:ER}
\end{figure}

Infine, all\rq interno del design \`{e} stato analizzato il controllo degli accessi. 
Il dettaglio completo delle attivit\`{a} svolte in questa fase \`{e} nel documento di System Design allegato.

\subsection{Implementazione}
La fase di implementazione ha lo scopo di creare un prodotto software in grado di soddisfare tutti i requisiti individuati nela fase di pianificazione. Nell'ambito del progetto, tale attivit\`{a} viene svolta seguendo un modello evolutivo a prototipi. 

Il primo prototipo implementato ha lo scopo di comprendere al meglio i requisiti del web service e del portale web. In questa fase non \`{e} stato ritenuto necessario implementare il prototipo dell\rq applicazione mobile in quanto essa presenta gli stessi requisiti del portale web. 

In particolare il percorso di sviluppo \`{e} stato il seguente:\\
- Il web service \`{e} stato implementato in linguaggio php, con l\rq utilizzo di un framework esterno, Slim, il cui scopo \`{e} quello di mappare gli url inseriti dall\rq utente in query che presentino le informazioni dal database. In questo modo viene garantito che il web service utilizzi il protocollo rest. I risultati sono forniti in formato JSON.(\emph{figura 18}) \\
- La web application \`{e} stata realizzata utilizzando Bootstrap [Aggiungere referenza] e il framework jQuery [Aggiungere referenza]. Nel primo prototipo sono state implementate soltanto le funzionalit\`{a} principali, tralasciando la gestione dell\rq utente e degli avvisi.(\emph{figura 19}) \\

\begin{figure*}[tb]
\centering
\includegraphics[scale=.3]{img/19.png}
\caption{Esempio risultato ricerca linea su Web service }
\label{fig:mhs}
\end{figure*} 

\begin{figure*}[tb]
\centering
\includegraphics[scale=.3]{img/20.png}
\caption{Esempio risultato ricerca linea su Web application }
\label{fig:mhs}
\end{figure*} 


Analisi:\\
Il primo prototipo ha evidenziato di essere inefficiente in quanto alcune query rilevanti hanno mostrato tempi di risposta troppo elevati. A tal proposito sono state valutate diverse alternative, tra cui una diversa architettura della base di dati e l\rq utilizzo di un database ad oggetti come MongoDB\footnote{http://www.mongodb.org/ }.\\
Il portale web ha chiarito il modo di visualizzazione delle informazioni per quanto riguarda la gestione delle fermate e la gestione delle linee. 



\subsection{Testing}
La pianificazione del testing prevede tre tipologie di test: testing di unit\`{a}, testing di integrazione e testing di sistema.\\
In merito al testing di unit\`{a}, il processo di testing per il progetto "GoBus" verr\`{a} sottoposto ad una tipologia di testing di tipo black-box, ovvero, si stresser\`{a} il sistema dal punto di vista funzionale, analizzando e confrontando gli output derivanti dagli input immessi, con gli output attesi. In combinazione con il testing black-box sar\`{a} utilizzato category partition. Questa tecnica permette di dividere i possibili input in gruppi, i cui elementi generano output logicamente simili. Tali gruppi, identificati come classi di equivalenza, saranno identificati e utilizzati in modo da eseguire almeno un test per ognuno di essi. Per agevolare il testing di regressione, in seguito alla modifica delle componenti software verr\`{a} utilizzato un framework adatto.\\
Relativamente al testing di integrazione, verr\`{a} adottata una strategia di tipo bottom-up. In sintesi si testeranno le componenti del sistema partendo dal layer pi\`{u} in basso nella gerarchia. L\rq implementazione di tale strategia prevede la costruire di driver test, ovvero di classi che utilizzino in maniera banale le componenti sotto test.\\
Il testing di sistema verr\`{a} effettuato mediante l\rq analisi empirica statica, ovvero confrontando il comportamento del sistema in esecuzione in relazione al comportamento atteso.\\

\begin{figure}[h]
\centering
\includegraphics[scale=.6]{img/21.png}
\caption{Esempio Test case }
\label{fig:mhs}
\end{figure}  

\begin{figure}[h]
\centering
\includegraphics[scale=.6]{img/22.png}
\caption{Esempio combinazione Test}
\label{fig:mhs}
\end{figure} 

\begin{figure}[h]
\centering
\includegraphics[scale=.6]{img/23.png}
\caption{Esempio category partition }
\label{fig:mhs}
\end{figure} 

\subsection{Rilascio}
Il prototipo viene rilasciato. Si pu\`{o} fare riferimento al video che mostra la demo dell\rq applicazione, (in allegato nel materiale consegnato) e al file zip contenente il codice dell\rq implementazione.



\section{Conclusioni e Sviluppi Futuri}
In conclusione, ripercorriamo tutti i passi del ciclo di vita del progetto, che ha previsto non solo un ciclo di vita ingegneristico, ma anche anche manageriale. Il primo passo \`{e} stato quello di adottare i 5 gruppi di processi descritti nel PMBOK: \emph{initiating, planning, executing, monitoring and controlling, closing}.\\
Il documento di Statement of Work \`{e} stato il punto di partenza del progetto. Nel documento, stilato insieme agli stakeholder, sono stati inclusi i \emph{Business Needs}, ovvero le esigenze di business. Ad ognuna di queste \`{e} stato associato un piano di risoluzione.
Si \`{e} quindi proceduto con la fase di pianificazione, definendo il documento principale, ovvero il \emph{Software Project Management Plan} (SPMP). Questo documento include tutte le informazioni provenienti dalle dieci aree di conoscenza del project manangement. Particolarmente rilevanti, sono stati il \emph{Requirements Management Plan}, per la gestione dei requisiti, il \emph{Software Configuration Management Plan} per l\rq implementazione del Configuration Management, il \emph{Quality Plan} e il \emph{Risk Management Plan}.\\
Molti documenti sono stati scritti in contemporanea e alcuni hanno subito diverse revisioni (ad esempio il SPMP).
A partire dal \emph{Requirements Management Plan} \`{e} stato quindi possibile effettuare la raccolta requisiti e l\rq analisi degli stessi. Ovviamente molti di questi requisiti sono stati sviluppati a partire dalle business needs iniziali. 
Nella fase di planning si \`{e} sono state quindi pianificate le fasi di esecuzione, monitoraggio e controllo.\\
Durante la fase di esecuzione sono stati prodotti documenti fondamentali per quanto concerne il processo d\rq Ingegneria del Software quali il System Design Document (SDD) e l\rq Object Design Document (ODD) sviluppati partendo dal documento di analisi dei requisiti.
Allo stato attuale, la fase di implementazione \`{e} stata considerata un\rq attivit\`{a} secondaria rispetto alla produzione di documentazione, in quanto ci si \`{e} limitati alla sola creazione di un prototipo di web service e di application web, allegato alla documentazione.
Nella fase di chiusura, che avverr\`{a} a fine progetto, ci sar\`{a} la release dei prodotti con conseguente post-mortem review in cui saranno specificate le best practices adottate e le lessons learned.\\
I prossimi passi da fare riguarderanno:
\begin{itemize}
\item l\rq implementazione dell\rq applicazione mobile per la piattaforma Windows Phone;
\item l\rq applicazione web,a supporto delle agenzie di trasporti;
\item l\rq implementazione completa del web service attraverso un protocollo di comunicazione restful. Il servizio sar\`{a} implementato usando tecnologie pi\`{u} performanti, quali ad esempio un DBMS ad oggetti.
\end{itemize}
\begin{thebibliography}{1}

\bibitem{PMBOK}
Project Management Institute. (2004). A guide to the project management body of knowledge (PMBOK guide). Newtown Square, Pa, Project Management Institute.

\bibitem{ITIL}
It Service Management: A Guide for Itil Foundation Exam Candidates (Second Edition) di Ernest Brewster, Richard Griffiths, Aidan Lawes, John Sansbury.


\end{thebibliography}




% that's all folks
\end{document}


