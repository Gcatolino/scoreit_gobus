Dal punto di vista organizzativo del progetto \emph{GoBus}, si è deciso grazie alle esperienze acquisite dal team di cercare di trovare un giusto equilibrio tra il ciclo di vita di un progetto, costituito dalle fasi di ingegnerizzazione e produzione, e il ciclo gestionale costituito invece dalle attività e processi svolti dal team per governare il progetto. Quindi questa fusione di conoscenze ci ha permesso di gestire al meglio il ciclo di vita del progetto, si è deciso di seguire le linee guida dettate dal \emph{Project Management Body of Knowledge} che descrive l’insieme delle prassi standard per la gestione di progetti così come definite dal “Project Management Institute”. I processi del Project Management descritti nel \emph{PMBOK®} \cite{PMBOK} sono 47 distribuiti in 5 gruppi, in particolare nell’ambito di GoBus sono stati utilizzati per:

\begin{itemize}
	\item {\bf{Inizializzazione}}: Raggruppa i processi necessari a selezionare un progetto in relazione a specifici obiettivi di business e definendo obiettivi e modalità di gestione del progetto.
	\item {\bf{Pianificazione}}: Finalizzata a circoscrivere l’ambito ed i deliverables del progetto, a definire requisiti di ciascun deliverable e a definire un piano di Project Management contenente tutti i vari piani in dettaglio per la gestione di: tempi, risorse, costi, qualità, rischi, comunicazione.
	\item {\bf{Esecuzione}}: Finalizzata a gestire e sviluppare il progetto, produrre i deliverables concordati.
	\item  {\bf{Monitoraggio e Controllo}}: Servono a valutare l’avanzamento dei lavori, a gestire eventuali modifiche e a verificare la qualità di quanto realizzato.
	\item {\bf{Chiusura}}: Finalizzata a gestire la chiusura del progetto.
\end{itemize}

\subsection{Inizializzazione}
In questa fase è stato inizializzato il progetto puntando prima a definire il team e identificare gli stakeholder di progetto e successivamente a definire Business need e strategia da attuare. Per quanto riguarda il team è stato creato come deliverable il team contract dove sono state definite alcune regole che riguardano comunication e people management, mentre per gli stakeholder è stato creato un registro dove memorizzare i vari stakeholder del progetto nel tempo e i loro contatti. Dopo aver creato le basi del team di progetto si è definito come lo Statement of work che racchiude in se le caratteristiche che dovrà avere il progetto, ovvero le Business needs, le funzionalità chiave e il piano strategico che si punta ad ottenere con GoBus. Per quanto riguarda i Bussiness needs come definito anche da ITIL® \cite{ITIL}, \emph{``ogni servizio deve avere una giustificazione''} per questo motivo all’interno di un progetto è fondamentale identificarli e renderli protagonisti della strategia di sviluppo del progetto. Tra i Business needs citati nello Statement of work abbiamo ad esempio la fruibilità dell’informazione, vitale è la knowledge sharing, ovvero la diffusione della conoscenza e informazione tramite il servizio offerto. Dallo Statement of work è già possibile iniziare a tirar fuori i requisiti funzionali che saranno successivamente identificati meglio all’interno del Requirements Analysis Document.

\subsection{Pianficazione}
TODO

\subsection{Esecuzione}
TODO

\subsection{Monitoraggio e Controllo}
Il monitoraggio e controllo è una fase che ha inizio insieme alla fase di pianificazione e dura fino alla fase di chiusura, è una fase dove si valuta l’avanzamento dei lavori e si attua il piano che è stato concepito nel Verification \& Validation Plan seguendo le metriche di misurazione di qualità definite nel Quality Plan. In pratica durante le fasi di pianificazione e esecuzione tramite l’utilizzo di checklist definite nel V \& V si revisionano i documenti creati. Vi sono diverse tipologie di checklist quelle per i documenti, dove vi è un componente del team che compila le checklist e quando termina invia le checklist al creatore del documento che la controlla e decide di accettarla e apportare le modifiche richieste. Di seguito il template utilizzato per la checklist di revisione dei documenti:\\

\begin{figure*}[!h]
\centering
\includegraphics[scale=.6]{img/checklist.png}
\caption{Checklist valutazione generica documento}
\end{figure*}

 Altre tipologie di checklist che abbiamo definito sono per la raccolta e analisi dei requisiti, stima dei costi e dei tempi, software configuration management, risk management, quality assurance e  testing. La compilazione di queste checklist da parte del team porta a una chiara misurazione della qualità del progetto, per maggiori dettagli sulle checklist invito a consultare il documento Verification \& Validation Plan. Dopo questa analisi che ci ha permesso di monitorare il progetto sono stati redatti dei management e quality report per capire appunto cosa doveva essere revisionato e aggiornato, un esempio nel quality report:\\
 
\begin{figure}[!h]
\centering
\resizebox{0.83\linewidth}{!}{\includegraphics[scale=.5]{img/risultato.png}}
\caption{Risultato Checklist specifica}
\end{figure}

Dove B sta a indicare che il documento è stato accettato ma vanno apportate lievi modifiche.

\subsection{Chiusura}
Una volta che arriveremo alla fase di chiusura di progetto e quindi aver ottenuto il risultato sulla base delle specifiche date in input, si procederà con la stesura della post-mortem review ovvero un deliverable per analizzare gli elementi del progetto e capire se sono stati di successo o di insuccesso. Template che sarà utilizzato per la post-mortem Review:\\

\begin{figure*}[!h]
\centering
\includegraphics[scale=.6]{img/postmortem.png}
\caption{Tabella valutazione finale}
\end{figure*}

Dopo aver dato il proprio giudizio vi sono una serie di domande aperte che riguardano lesson learned e considerazioni finali.
La post-mortem review ci aiuterà nel gestire futuri rischi già incotrati in questo progetto quindi migliorando le nostre skill dal punto di vista del risk management e delle Best Practice.
