La maggior parte delle citt\`{a}, grandi o medie che siano, posseggono una fitta rete di collegamenti. Tali collegamenti il più delle volte appaiono difficili da comprendere e creano spesso confusione sia non solo ai visitatori occasionali, ma anche alle persone più pratiche della citt\`{a}. Nel corso degli anni, il progresso tecnologico ha fatto in modo che fossero sviluppate applicazioni mobile a supporto della fruibilit\`{a} delle informazioni sui collegamenti che le aziende di trasporto forniscono. 
Da allora i più famosi store per smartphone (App Store\footnote{https://itunes.apple.com/us/genre/ios/id36?mt=8}, Google Play \footnote{https://play.google.com/store}, Windows Phone\footnote{http://www.windowsphone.com/it-it/store}) rendono disponibili ogni giorno e con continui aggiornamenti applicazioni mobile, ma queste, per la maggior parte, sono legate solamente alla specifica città e non orientate verso un concetto più ampio come l’intera nazione.\\ 
Avere un’applicazione di carattere nazionale consentirebbe a tutti i cittadini o turisti di essere maggiormente informati su tutta la mobilit\`{a} italiana. Risulterrebbe inoltre pi\`{u} pratica dato che in una sola \emph{app} potrebbero essere consultati e/o programmati orari, linee e percorsi di tutte le corse urbane della propria città o a seconda di dove ci si trova.\\
\emph{Moovit} \footnote{https://itunes.apple.com/it/app/moovit-trasporto-pubblico/id498477945?mt=8} (disponibile per i tre principali store) è una delle poche applicazioni che comprende una vasta scelta di città (all'incirca 20). Rende non solo disponibile informazioni di collegamento riguardo autobus ma anche di treni e metro (qualora presenti). L'applicazione consente di ricercare e di fruire informazioni riguardo ogni tipo di collegamento e pianificare un eventuale viaggio urbano. 
La caratteristica più interessante di questa applicazione è che risulta molto social. Essa infatti reperisce informazioni sulla qualità del percorso della linea di un collegamento (affollamento, orario di arrivo) tramite le valutazione degli utenti. 
Nel suo complesso \emph{Moovit} risulta essere completa e molto utile, in più viene resa appetibile grazie al suo aspetto social, come sopra citato. Tuttavia presenta dei limiti. Navigandola infatti, si mostra inizialmente abbastanza confusionaria in quanto risulta complicato capire come poter usufruire delle funzionalità che essa mette a disposizione. Un altro limite risiede nella sola possibilità di basarsi sulla posizione attuale, escludendo quindi la possibilità di selezionare una diversa città, qualora si voglia prevenire futuri spostamenti in altre metropoli.
Inoltre è ovvio pensare che il servizio \emph{``Social''} (ciliegina sulla torta  dell'applicazione) non potrà essere garantito per sempre, poichè troppo legato all'attività degli utenti.\\
Da questa problematica nasce l’idea di sviluppare \emph{GoBus}, un’applicazione pensata per essere accessibile sia tramite una web application che da dispositivi mobile (ad esempio, Windows Phone).\\
\emph{GoBus} consentirà di avere a disposizione tutti i collegamenti delle citt\`{a} italiane, a seconda di dove ci si trova. Consentirà di dare informazioni sui percorsi da seguire, e in base alla propria posizione visualizzare fermate e linee nelle vicinanze, oppure scegliere comodante da soli una citt\`{a}.
Tutto questo grazie anche alle aziende di trasporti (oltre agli admin dell’applicazione) le quali potranno caricare i dettagli appartenenti ai collegamenti e le varie news associate. Tali file che le aziende caricheranno non saranno lavoro aggiuntivo per loro, dato che si utilizzeranno gli stessi file che le aziende forniscono a Google.