La maggior parte delle citt\`{a}, grandi o medie che siano, posseggono una fitta rete di collegamenti. Tali collegamenti il più delle volte appaiono difficili da comprendere e creano spesso confusione sia non solo ai visitatori occasionali, ma anche alle persone più pratiche della citt\`{a}. Nel corso degli anni, il progresso tecnologico ha fatto in modo che fossero sviluppate applicazioni mobile a supporto della fruibilit\`{a} delle informazioni sui collegamenti che le aziende di trasporto forniscono, ma queste sono legate solamente alla specifica città e non orientate verso un concetto più ampio come l’intera nazione. Avere un’applicazione di carattere nazionale consentirebbe a tutti i cittadini o turisti di essere maggiormente informati su tutta la mobilit\`{a} italiana. Sarebbe inoltre più pratica dato che in una sola app potrebbero essere consultati e/o programmati orari, linee e percorsi di tutte le corse urbane della propria città o a seconda di dove ci si trova. Da questa problematica nasce l’idea di sviluppare \emph{GoBus}, un’applicazione pensata per essere accessibile sia tramite una web application che da dispositivi mobile (ad esempio, Windows Phone).\\

GoBus consentirà di avere a disposizione tutti i collegamenti delle citt\`{a} italiane, a seconda di dove ci si trova. Consentirà di dare informazioni sui percorsi da seguire, e in base alla propria posizione visualizzare fermate e linee nelle vicinanze, oppure scegliere comodante da soli una citt\`{a}.