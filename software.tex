Questa sezione descrive il processo di sviluppo adottato nell'ambito del profetto \emph{GoBus}.

\subsection{Modello del Ciclo di Vita}
In base alle esigenze relative alle tempistiche e agli obiettivi del progetto, il modello del ciclo di vita che meglio si adatta \`{e} il ciclo \emph{evolutivo a prototipazione}. In particolare, vista la necessita di comprendere meglio i requisiti che il committente richiedeva, la tecnica utilizzata consente di fornirgli un prototipo del sistema in tempi rapidi. In questo modo, \`{e} possibile periodicamente avere feedback utili per la continuazione delle attivit\`{a} progettuali.


Software: - Descrizione software ideale (app + portale web) - Descrizione di ciò che abbiamo un portale web che utilizza un web service che grazie al protocollo rest da delle informazioni… il prototipo ha queste funzionalità implementate: gestione linee e delle fermate…gestione percorso….

\subsection{Requirement Elicitation and Analysis}
La fase di requirement elicitation consiste nella comprensione delle necessit\`{a} del committente, al fine di collezionare i requisiti del sistema. Nell'ambito del progetto, tale attivit\`{a} \`{e} stata svolta facendo brainstorming con tutti gli stakeholder interessati. In particolare, sono stati organizzati incontri sia con i coordinatori del progetto sia con aziende di trasporto locali. Per mitigare il richio di errata comprensione delle reali necessit\`{a} degli stakeholder, i meeting si sono ripetuti così che, ad ogni incontro, i requisiti potessero essere meglio fissati. Il numero totale di incontri con gli stakeholder \`{e} 3.\\
La tabella \ref{tab:gestioni} mostra, ad alto livello, i principali raggruppamenti delle funzionalità identificate nell'ambito del progetto \emph{GoBus}. Di seguito verranno analizzate in dettaglio ciascun raggruppamento.\\

\begin{table*}[tb]
   \centering
   \caption{Overview delle Funzionalit\`{a} Identificate}
   \label{tab:gestioni}
  \resizebox{1\linewidth}{!}{
   \begin{tabular}{ll}\hline
   Nome & Descrizione\\\hline
   Gestione Registrazione & Insieme di funzionalit\`{a} che consentono la registrazione di un nuovo utente.\\
   Gestione Autenticazione & Insieme di funzionalit\`{a} che consente il riconoscimento degli utenti registrati.\\
   Gestione Account & Insieme di funzionalit\`{a} per la manipolazione degli account degli utenti.\\
   Gestione Fermate & Funzionalit\`{a} che consente di visualizzare le fermate tramite l’applicazione di diversi filtri per la ricerca mirata.\\
   Gestione Linee & Funzionalit\`{a} che consente la visualizzazione delle linee tramite l’applicazione di diversi filtri per la ricerca mirata.\\
   Gestione Percorso & Insieme di funzionalit\`{a} che consente la visualizzazione delle indicazioni stradali.\\
   Gestione Preferiti & Funzionalit\`{a} che consente la gestione delle fermate e delle linee preferite.\\
   Gestione News & Insieme di funzionalit\`{a} per la efficace gestione degli avvisi da parte delle aziende di trasporti.\\
   Gestione Dati GTFS & Insieme di funzionalit\`{a} che consente ad un’azienda di trasporti di caricare il proprio file GTFS.\\
   \hline
   \end{tabular}
   }
\end{table*}

\noindent {\bf{Gestione Registrazione}}: La gestione della registrazione da la possibilità ad un nuovo utente di iscrizione alla piattaforma. In particolare, i requisiti funzionali delineati sono:\\

\indent {\bf{RF-0.1}} - \emph{Registrazione tramite Email}: Consente a un utente di registrarsi utilizzando come dati un email e password.\\
\indent {\bf{RF-0.2}} - \emph{Registrazione tramite Facebook}: Consente a un utente di registrarsi utilizzando i dati di facebook.\\
\indent {\bf{RF-0.3}} - \emph{Registrazione tramite Google}: Consente a un utente di registrarsi utilizzando i dati di Google.\\

Vale la pena notare che questo insieme di requisiti \`{e} comune sia alla parte relativa alla web application che all'applicazione mobile. Per evitare ridondanze, i requisiti sono stati riportati una sola volta.\\

\noindent {\bf{Gestione Autenticazione}}: Questa gestione permette ad un utente registrato di essere riconosciuto e abilitato a effettuare determinate operazioni da parte del sistema:\\

\indent {\bf{RF-1.1}} - \emph{Login}: Consente ad un utente registrato al sistema di autenticarsi e svolgere operazioni per la sua area di competenza.\\
\indent {\bf{RF-1.2}} - \emph{Login}: Consente all’utente di uscire dal sistema.\\

\noindent {\bf{Gestione Account}}: Questo insieme di funzionalit\`{a} si occupa di tutto ciò che ha a che fare con la manipolazione delle informazioni personali degli utenti registrati al sistema. Nel dettaglio:\\

\indent {\bf{RF-2.1}} - \emph{Visualizzazione Account}: Consente all'utente registrato (utente generico o azienda di trasporti) di visualizzare i dati relativi all’account personale.\\
\indent {\bf{RF-2.2}} - \emph{Modifica Account}: Permette all’utente generico e all’azienda di modificare i propri dati nel sistema.\\
\indent {\bf{RF-2.3}} - \emph{Eliminazione Account}: Consente ad un utente registrato di poter eliminare la propria iscrizione dal sistema.\\

\noindent {\bf{Gestione Fermate}}: Questa funzionalit\`{a} permette di visualizzare le fermate tramite l’applicazione di diversi filtri, al fine di effettuare una ricerca mirata delle informazioni di cui l’utente necessita. I requisiti funzionali identificati sono i seguenti:\\

\indent {\bf{RF-3.1}} - \emph{Visualizza Fermate tramite Localizzazione}: Consente all’utente di visualizzare una mappa con tutte le fermate presenti nelle vicinanze.\\
\indent {\bf{RF-3.2}} - \emph{Visualizza Fermate tramite Localit\`{a}}: Consente all’utente di visualizzare una lista di tutte le fermate presenti nella località selezionata.\\
\indent {\bf{RF-3.3}} - \emph{Visualizza Fermate tramite Nome}: Consente all’utente di effettuare una ricerca di una fermata specifica.\\
\indent {\bf{RF-3.4}} - \emph{Visualizza Linee relative ad una Fermata}: Consente all’utente di visualizzare le linee presenti in una fermata selezionata, cioè le linee che hanno quella fermata lungo la corsa.\\

\noindent {\bf{Gestione Linee}}: Questa funzionalit\`{a} permette di visualizzare le linee tramite l’applicazione di diversi filtri per effettuare una ricerca mirata delle informazioni di cui l’utente necessita:\\

\indent {\bf{RF-4.1}} - \emph{Visualizza Fermate tramite Localizzazione}: Consente all’utente di visualizzare una mappa con tutte le fermate presenti nelle vicinanze.\\
\indent {\bf{RF-4.2}} - \emph{Visualizza Fermate tramite Localit\`{a}}: Consente all’utente di visualizzare una lista di tutte le fermate presenti nella località selezionata.\\
\indent {\bf{RF-4.3}} - \emph{Visualizza Fermate tramite Nome}: Consente all’utente di effettuare una ricerca di una fermata specifica.\\
\indent {\bf{RF-4.4}} - \emph{Visualizza Linee relative ad una Fermata}: Consente all’utente di visualizzare le linee presenti in una fermata selezionata, cioè le linee che hanno quella fermata lungo la corsa.\\
\indent {\bf{RF-4.5}} - \emph{Visualizza Linee relative ad una Fermata}: Consente all’utente di visualizzare le linee presenti in una fermata selezionata, cioè le linee che hanno quella fermata lungo la corsa.\\
\indent {\bf{RF-4.6}} - \emph{Visualizza Linee relative ad una Fermata}: Consente all’utente di visualizzare le linee presenti in una fermata selezionata, cioè le linee che hanno quella fermata lungo la corsa.\\
\indent {\bf{RF-4.7}} - \emph{Visualizza Linee relative ad una Fermata}: Consente all’utente di visualizzare le linee presenti in una fermata selezionata, cioè le linee che hanno quella fermata lungo la corsa.\\


\subsection{Design}

\subsection{Implementazione}

\subsection{Testing}

\subsection{Rilascio}