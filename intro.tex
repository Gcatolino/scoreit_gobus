La moibilit\`{a} sostenibile \`{e} uno degli argomenti pi\`{u} importanti e dibattuti su cui l'Unione Europea, e tutto il mondo in generale, si sta focalizzando negli ultimi anni. L'abuso dei mezzi privati genera inquinamento acustico e atmosferico di una certa rilevanza, \`{e} necessario affrontare il problema a livelo locale, urbano ed interurbano. 

Tra le azioni attualmente intraprese per una maggiore sostenibilità della mobilità, gli enti pubblici si sono impegnati in una fitta e forte promozione di metodi alternativi per viaggiare, incentivando l'uso di biciclette, mezzi pubblici, mezzi privati in condivisione. Sono infatti note le iniziative nazionali intraprese nel corso degli anni, come ad esempio le "targhe alternate" imposte in passato nelle citt\`{a} italiane con maggiore inquinamento dovuto alla mobilità o la più recente "area C" promossa dalla città di Milano. \`{E} necessario, quindi, migliorare il sistema dei trasporti pubblici per migliorare la qualità della vita dei cittadini. Bisogna che gli enti pubblici locali, nazionali ed internazionali diffondano una vera e propria cultura della mobilità sostenivile.La stessa Unione Europea ha indetto un evento annuale dedicato all'argomento in questione: la "Settimana Europea della Mobilità Sostenibile". In questa occasione, gli enti pubblici promuovono giornate ed eventi atti alla sensibilit\`{a} dei cittadini sul problema dell'inquinamento.

In questo contesto, il nostro contibuto vuole essere di aiuto al miglioramento della vita dei cittadini attravero una più facile e rapida consultazione delle informazioni territoriali relative al trasporto pubblico. Un aggiornamento costante di quelle che sono le fermate e le corse di interesse dell'utente. Un viaggio in autobus non deve più essere un viaggio da organizzare, ma un'opportunità da sfruttare sempre a portata di mano. 