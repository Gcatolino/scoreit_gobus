La mobilit\`{a} sostenibile \`{e} uno degli argomenti pi\`{u} importanti e dibattuti su cui l\rq Unione Europea, e tutto il mondo in generale, si sta focalizzando negli ultimi anni. L\rq abuso dei mezzi privati genera inquinamento acustico e atmosferico, con conseguenti ricadute sulla salute dei cittadini. \`{E} quindi un problema serio, che \`{e} necessario a livello locale, urbano ed interurbano. 

Tra le azioni attualmente intraprese per una maggiore sostenibilit\`{a} della mobilit\`{a}, gli enti pubblici si sono impegnati in una fitta e forte promozione dei metodi alternativi per viaggiare, incentivando l\rq uso di biciclette, mezzi pubblici, mezzi privati in condivisione. 

Nell\rq ultima decade, una serie di iniziative sono state intraprese per arginare il problema dell\rq inquinamento. Ad esempio, ricordiamo l\rq utilizzo delle ``targhe alterne'' imposte nelle principali e maggiormente inquinate citt\`{a} italiane, o anche la pi\`{u} recente \emph{``area C''}, promossa dal comune di Milano, ovvero una zona, nel centro cittadino, in cui \`{e} stata introdotta una tassa proporzionale alla quantit\`{a} di inquinamento prodotto dal veicolo entrante nell\rq area.

Tutte queste iniziative rendono quindi necessario un miglioramento del sistema dei trasporti pubblici, al fine di migliorare la qualit\`{a} della vita dei cittadini diffondendo una vera e propria cultura della mobilit\`{a} sostenibile.

Gli enti pubblici hanno il compito di promuovere comportamenti ecosostenibili, come ad esempio la ``Settimana Europea della Mobilit\`{a} Sostenibile'', promosso dall\rq Unione Europea e dedicata alla sensibilit\`{a} dei cittadini sul problema dell\rq inquinamento.

In questo contesto, il nostro contibuto vuole essere di aiuto al miglioramento della vita dei cittadini attraverso una pi\`{u} facile e rapida consultazione delle informazioni territoriali relative al trasporto pubblico. L\rq idea alla base di questo progetto pu\`{o} essere riassunta nel seguente slogan:

\begin{center}
\emph{Un viaggio in autobus non deve pi\`{u} essere un viaggio da organizzare, ma un\rq opportunit\`{a} da sfruttare sempre a portata di mano.}
\end{center}

In questo documento sono descritti i processi di management attraverso i quali si intende portare a termine il progetto, oltre che la descrizione del processo di sviluppo adottato per la produzione del software. 

Il successo di un progetto \`{e} dato dal raggiungimento degli obiettivi richiesti dal sistema, entro i vincoli di scope, tempo e costi. \`{E} quindi importante che l\rq intera vita del progetto, dal suo concepimento alla sua chiusura, sia guidata da un processo di management ben definito. Per questo progetto \`{e} stato preso in considerazione il PMBOK \cite{PMBOK}, uno standard \emph{de facto} in cui \`{e} descritto l\rq insieme dei processi che portano al successo di un progetto software. Essi sono concettualmente divisi per aree di conoscenza, cio\`{e} competenze chiave che un manager deve avere per poter gestire un progetto, e gruppi di processi che identificano in quali fasi del ciclo di vita questi processi vengono mappati.\\
Basandosi su tali concetti, \`{e} stata messa in piedi una struttura organizzativa che consentisse un\rq efficace ed efficiente gestione e controllo delle attivit\`{a} che, via via, hanno trasformato l\rq idea iniziale in un prodotto realmente fruibile.\\Il prototipo realizzato a valle del processo di progettazione, consiste di un web service capace di interrogare i dati messi a disposizione dalle aziende di trasporto al fine di fornire le informazioni sulla mobilit\`{a}. Una applicazione web \`{e} stata costruita per dimostrare la reale utilit\`{a} di tali informazioni

{\bf{Struttura del documento:}} La sezione II descrive il contesto in cui trova spazio il nostro lavoro, definendo il problem statement; La Sezione III riepiloga l\rq insieme dei processi manageriali applicati, mentre la sezione IV descrive i processi tecnici che fanno riferimento alla progettazione del prototipo realizzato. Infine, sono proposte alcune considerazioni sul lavoro effettuato ed alcuni interventi da effettuare in futuro. 