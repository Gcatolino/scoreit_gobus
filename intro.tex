La mobilit\`{a} sostenibile \`{e} uno degli argomenti pi\`{u} importanti e dibattuti su cui l\rq Unione Europea, e tutto il mondo in generale, si sta focalizzando negli ultimi anni. L\rq abuso dei mezzi privati genera inquinamento acustico e atmosferico ed \`{e} quindi necessario affrontare il problema a livello locale, urbano ed interurbano. 

Tra le azioni attualmente intraprese per una maggiore sostenibilit\`{a} della mobilit\`{a}, gli enti pubblici si sono impegnati in una fitta e forte promozione dei metodi alternativi per viaggiare, incentivando l'uso di biciclette, mezzi pubblici, mezzi privati in condivisione. 

Al fine di risolvere i problemi causati dall\rq abuso di mezzi privati sono state prese nel tempo iniziative, tra queste citiamo ad esempio le ``targhe alternate'' imposte in passato nelle città italiane con maggiore inquinamento dovuto alla mobilità o la più recente ``area C'' promossa dal comune di Milano. \`{E} necessario, quindi, migliorare il sistema dei trasporti pubblici per migliorare la qualit\`{a} della vita dei cittadini diffondendo una vera e propria cultura della mobilit\`{a} sostenibile.

Gli enti pubblici hanno il compito di promuovere comportamenti ecosostenibili, come ad esempio la ``Settimana Europea della Mobilità Sostenibile'', promosso dall'Unione Europea e dedicata alla sensibilit\`{a} dei cittadini sul problema dell\rq inquinamento.

In questo contesto, il nostro contibuto vuole essere di aiuto al miglioramento della vita dei cittadini attraverso una più facile e rapida consultazione delle informazioni territoriali relative al trasporto pubblico. Un viaggio in autobus non deve più essere un viaggio da organizzare, ma un'opportunità da sfruttare sempre a portata di mano. 

In questo documento sono descritti i processi di management attraverso i quali si intende portare a termine il progetto e la descrizione delle fasi del ciclo di vita del software da produrre. 

Il successo di un progetto \`{e} dato dal raggiungimento degli obiettivi richiesti dal sistema, entro i vincoli di scope, tempo e costi. E' quindi importante che l'intera vita del progetto, dal suo concepimento alla sua chiusura, sia guidata da un processo di management ben definito. Per questo progetto è stato preso in considerazione il PMBOK, "A Guide to the Project Management Body of Knowledge", una guida in cui sono contenuti i processi che portano un progetto al successo. Essi sono concettualmente divisi per aree di conoscenza, cio\`{e} competenze chiave che un manager deve avere per poter gestire un progetto, e gruppi di processi che identificano in quali fasi del ciclo di vita questi processi vengono mappati. 

Nello specifico, le aree di conoscenza identificate dalla guida sono: la gestione dell\rq integrazione di tutte le altre aree, la gestione dello scope, la gestione del tempo, la gestione dei costi, la gestione della qualit\`{a}, la gestione delle risorse umane, la gestione della comunicazione, la gestione dei rischi, la gestione dell\rq approvvigionamento e la gestione degli stakeholder. 

I gruppi di processi invece sono identificati in: avvio, programmazione, esecuzione, controllo e chiusura. 

\`{E} importante soffermarsi soprattutto sui processi di programmazione e controllo, in quanto descrivono in che modo si intende procedere nella gestione del progetto e un costante controllo dei risultati ottenuti rispetto agli obiettivi pianificati. In questo documento sono riassunti i deliverable principali, quali output dei processi seguiti.

\`{E} inoltre presente i) una descrizione del software da realizzare, che comprende un web service, una web application e un'applicazione mobile per la piattaforma Windows Phone, ii) una descrizione dettagliata di come si intende procedere per implementarlo, testarlo, rilasciarlo, iii) alcune riflessioni sui possibili sviluppi futuri.